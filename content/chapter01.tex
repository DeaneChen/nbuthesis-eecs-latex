
\chapter{背景与意义} 

\section{选题背景} 

图像超分辨率(SR)\cite{kim2010csmacdwithreservations,luguojun2019jiyumqtt}是一种在计算机视觉领域的图像生成技术,是一种将现有图像通过一定的方式提升分辨率大小的计算过程。提高分辨率能够让图像包含更多的信息,在医疗、工业、商业等诸多领域都有广泛的应用前景 。

……

……

\vspace*{\baselineskip}

\section{相关研究工作}

自Keys R.于1981年提出“双三次插值法”\cite{songqi2020huaweiharmonyos}以来,图像超分辨率重建一直是计算机图像领域的一个研究热点。目前,超分辨率图像重建工作主要沿着插值法、成像模型重建和机器学习这三大技术方向进行实现。同时,为了设计一个基于真实度的图像评价模型,本节还将介绍几种经典的全参考图像质量评价算法。

\subsection{基于插值法的图像超分辨率重建算法} 

基于插值法的图像超分辨率算法主要针对已有的输入数据进行工作,通过对临近像素信息的获取和分析,拟合出能够填入临近像素间的新像素的成像信息。在保证图像整体风格不变的基础上,该方法能提高图像分辨率,增强图像可用性和可识别性。

……

……

\vspace*{\baselineskip} 

\subsection{基于成像模型重建的超分辨率算法}

基于重建模型的超分辨率算法通过将图像的一些基本概念和成像规律作为先验加入到超分辨率重建过程中,使得图像的超分辨率重建从一个随机性问题变为有先验约束的求确定解的问题。

……

……

基于重建模型的超分辨率算法通过将图像的一些基本概念和成像规律作为先验加入到超分辨率重建过程中,使得图像的超分辨率重建从一个随机性问题变为有先验约束的求确定解的问题。

基于重建模型的超分辨率算法通过将图像的一些基本概念和成像规律作为先验加入到超分辨率重建过程中,使得图像的超分辨率重建从一个随机性问题变为有先验约束的求确定解的问题。



\vspace*{\baselineskip} 

\section{论文结构安排}

本文共五章,论文后续内容的结构安排如下:

相关理论知识(第二章)

基于对抗生成的图像超分辨率算法(第三章)

基于人眼视觉感知的图像评价模型(第四章)

总结和展望(第五章)

第二章介绍了SRGAN算法提出的动机和背景以及具体的算法设计,并对网络的主要工作原理进行了分析。另外,介绍了两种生成对抗模型ESRGAN和WGAN,概括说明了两种模型的创新点和部分结构。其中详细介绍了Wasserstein距离并对带有Wasserstein距离的感知函数单元进行了推导和功能分析。最后结合公式和图像介绍了几种经典的图像评价指标,包括MSE值、PSNR值、SSIM值以及主观真实性评价等。


第三章首先分析了当下对抗生成网络存在的问题。……

第四章提出了一种新的基于人眼视觉感知的图像评价算法,用来弥补如今超分辨率重建领域基于真实性的评价算法较为欠缺的问题。本文通过增加融合亮度层提高模型对失真图像缺失边缘的感知,减弱传统梯度算法对平滑区域过于强调的效果。……

第五章对全文工作进行总结的同时,也对超分辨率重建模型的优化方向和图像质量评价算法的优化路线方面做了展望。